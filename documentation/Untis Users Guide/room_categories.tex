
\section{Raumkategorien}
\label{sec:rc_list}
\begin{description}
	\item[Hörsäle] - Hörsäle haben feste Bestuhlung und meistens einen aufsteigender Boden.
	\begin{description}
		\item[HS.A] Hörsaal, Auditorium (200 oder mehr Sitzplätze)
		\item[HS.G] Hörsaal, Groß (90 bis 199 Sitzplätze)
		\item[HS.M] Hörsaal, Mittel (70 bis 89 Sitzplätze)
		\item[HS.K] Hörsaal, Groß (bis 69 Sitzplätze)
	\end{description}
	\item[Fachsäle] - Hörsäle die besondere Schwerpunkte gewidmet sind. Sie haben feste Bestühlung und meistens einen aufsteigender Boden.
	\begin{description}
		\item[FS.C] Fachsaal, Chemie
		\item[FS.P] Fachsaal, Physik
	\end{description}
	\item[Seminarräume] - Seminarräume haben lose Bestuhlung und einen flachen Boden.
	\begin{description}
		\item[SR.G] Seminarraum, Groß (60 oder mehr Sitzplätze)
		\item[SR.M] Seminarraum, Mittel (20 bis 59 Sitzplätze)
		\item[SR.K] Seminarraum, Groß (bis 19 Sitzplätze)
	\end{description}
	\item[LAR]	Laborrräume - besondere Schwerpunkte werden als im Feld \texttt{Text} eingetragen.
	\item[PCL]	PC Labore
	\item[GAR]	Gruppenarbeitsraum
	\item[BR] 	Büroraum
	\item[XRM]	Raum, nicht bekannt
\end{description}